\chapter{Signalverarbeitung bei ballistokardiographischen Signalen}

\section{Vorverarbeitung}

\section{Grundsätzliches}

\section{Detektion von Herzschlägen}

	\begin{itemize}
		\item in dieser Arbeit Fokus auf von Brüser entwickelter Algorithmus
		\item in \ref{ballistokardiographie} erwähnte Annahme: aufeinander folgende Herzschläge ähneln sich
	\end{itemize}

\section{Artefakterkennung}

	\begin{itemize}
		\item Artefakte = irrelevante Signalteile mit variierender Amplitude, Frequenz und Dauer, die physiologiches Signal stören \footcite{Nizami2013}
		\item Ziel Beurteilung Signalqualität: nur die Teile des Signals, die Vitalparameter enthalten verarbeiten
		\item Bewegungsartefakte, Sensorstörungen etc nicht verarbeiten
		\item Quelle Störung irrelevant, aben medizinische Abnormalitäten dürfen nicht als gestörtes Signal klassifiziert werden
		\item Signalqualität oft mit so genannten Signal Quality Indices gemessen
		\item je nach SQI und Anwendungsfall verschiedene Aussagen
		\item \citeauthor{Sadek2016}: Unterscheidung in Bezug auf Signalqualität zwischen informativ und nicht informativ
		\item informativ: noise und Signal von guter Qualität, Features können ohne weitere Verarbeitung extrahiert werden
		\item nicht informativ: Informationen mit Artefakten und noise vermischt, weitere Verarbeitung vor Extraktion Vitalparameter nötig oder Extraktion von physiologischen Eigenschaften unmöglich
		\item \footcite{HoogAntink2020} festgestellt, dass bei Messsystemen in Betten besonders die Messung während des Tages große Signalteile von schlechter Qualität aufweisen, mehr als Nachts

	\end{itemize}

	\subsection{Schwellwertbasierte Artefakterkennung}
	
	\subsection{Ähnlichkeit der Intervallschätzer von Brüser}
	
	\subsection{Maschinelles Lernen mit statistischen Merkmalen}
	
	Ein Algorithmus zur Beurteilung der Signalqualität mittels maschinellen Lernens wird von \citeauthor{Sadek2016} im Paper \citetitle{Sadek2016} beschrieben. Betrachtet werden \ac{BKG}-Signale, die in einem Massagesessel aufgenommen werden.

\section{Messdaten}
	
	\subsection{Erfassung}
	
	\begin{itemize}
		\item aufgenommen in der Gefäßstation des Universitätskrankenhauses in Tampere in Finnland
		\item 14 Patient*innen wurden bis zu 24 h überwacht
		\item 2 weiblich, 12 männlich
		\item Durchschnittsalter: 69,57 Jahre
		\item nach verschiedenen gefäßchirurgischen Eingriffen
		\item durchschnittliche Messdauer: 17.7 h, range 4,46 bis 22,96 h
		\item EMFit QS Bettsensor, zwischen Matratze des Krankenhausbettes und Bettgestehl positioniert
		\item Samplingrate des EMFit QS Systems: 100 Hz, Bandpass-limitiert auf 1 bis 5 Hz
		\item Referenz EKG: Faros 360 5 lead Holter monitor, 1 kHz Abtastrate
		\item variabler Drift zwischen beiden Signalen % TODO: näher beschreiben
	\end{itemize}
	
	\subsection{Vorliegende Form}
	
	\begin{itemize}
		\item unbearbeitetes \ac{BKG}-Signal, abgesehen von der Bandpasslimitierung auf 1 bis 5 Hz
		\item unbearbeitetes 3-Kanal EKG Signal
		\item mit CLIE-Algorithmus detektierte Herzschläge, schon nach Qualität gefiltert mitsamt Brüser SQI, Länge und Länge des Herzschlages der EKG Referenz
		\item Vektoren, die den Drift der beiden Signale beschreiben, Form Sekunde \ac{BKG}-Signal und entsprechende Sekunde in \ac{EKG}-Referenz
	\end{itemize}
	
	\subsection{Verarbeitung und Datenstruktur}
	
	\begin{itemize}
		\item Datensatz von einem Patient besteht aus BKG Signal und EKG Referenz
		\item beides wird eingelesen, geprüft ob schon Detektion von Herzschlägen (Erkennen von R-Peaks bzw. CLIE Algorithmus schon durchgeführt wurde und als csv-Datei existiert
	\end{itemize}

	\subsection{Annotation der Daten}
	
	Die vorliegenden Daten sind nicht annotiert. Es ist im Rahmen dieser Arbeit nicht möglich, die Annotation durch Expert*innen durchführen zu lassen, weshalb auf das parallel aufgenommene \ac{EKG} zurückgegriffen wird.
	
	\begin{itemize}
		\item aufgrund des nicht-linearen Drifts der Daten herzschlaggenaue Synchronisierung schwierig
		\item Entscheidung Annotation von Bereichen von mehren Sekunden möglich zu machen
		\item Ablauf: existiert EKG Signal zu diesem Zeitpunkt, bei dem eine Herzfrequenz ermittelt werden konnte
		\item Berechnung dieser -> Anzahl der R-Peaks in diesem Bereich -1 geteilt durch den Abstand des letzten und des ersten Peaks
		\item Berechnung \ac{BKG}-Herzfrequenz: Durchschnitt der geschätzten Längen der erkannten Peaks im Bereich
		\item Annotation anhand von relativer oder absoluter Abweichung der beiden
	\end{itemize}
	

