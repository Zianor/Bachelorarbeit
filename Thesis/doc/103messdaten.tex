\chapter{Signalverarbeitung bei ballistokardiographischen Signalen}

\section{Vorverarbeitung}

\section{Grundsätzliches}

\section{Detektion von Herzschlägen}

\section{Artefakterkennung}


	\subsection{Schwellwertbasierte Artefakterkennung}
	
	\subsection{Ähnlichkeit der Intervallschätzer von Brüser}
	
	\subsection{Maschinelles Lernen mit statistischen Merkmalen}
	
	Ein Algorithmus zur Beurteilung der Signalqualität mittels maschinellen Lernens wird von \citeauthor{Sadek2016} im Paper \citetitle{Sadek2016} beschrieben. Betrachtet werden \ac{BKG}-Signale, die in einem Massagesessel aufgenommen werden.

\section{Messdaten}
	
	\subsection{Erfassung}
	
	\begin{itemize}
		\item aufgenommen in der Gefäßstation des Universitätskrankenhauses in Tampere in Finnland
		\item 14 Patient*innen wurden bis zu 24 h überwacht
		\item 2 weiblich, 12 männlich
		\item Durchschnittsalter: 69,57 Jahre
		\item nach verschiedenen gefäßchirurgischen Eingriffen
		\item durchschnittliche Messdauer: 17.7 h, range 4,46 bis 22,96 h
		\item EMFit QS Bettsensor, zwischen Matratze des Krankenhausbettes und Bettgestehl positioniert
		\item Samplingrate des EMFit QS Systems: 100 Hz, Bandpass-limitiert auf 1 bis 5 Hz
		\item Referenz EKG: Faros 360 5 lead Holter monitor, 1 kHz Abtastrate
		\item variabler Drift zwischen beiden Signalen % TODO: näher beschreiben
	\end{itemize}
	
	\subsection{Vorliegende Form}
	
	\begin{itemize}
		\item unbearbeitetes \ac{BKG}-Signal, abgesehen von der Bandpasslimitierung auf 1 bis 5 Hz
		\item unbearbeitetes 3-Kanal EKG Signal
		\item mit CLIE-Algorithmus detektierte Herzschläge, schon nach Qualität gefiltert mitsamt Brüser SQI, Länge und Länge des Herzschlages der EKG Referenz
		\item Vektoren, die den Drift der beiden Signale beschreiben, Form Sekunde \ac{BKG}-Signal und entsprechende Sekunde in \ac{EKG}-Referenz
	\end{itemize}
	
	\subsection{Verarbeitung und Datenstruktur}
	
	\begin{itemize}
		\item Datensatz von einem Patient besteht aus BKG Signal und EKG Referenz
		\item beides wird eingelesen, geprüft ob schon Detektion von Herzschlägen (Erkennen von R-Peaks bzw. CLIE Algorithmus schon durchgeführt wurde und als csv-Datei existiert
	\end{itemize}

	\subsection{Annotation der Daten}

Die vorliegenden Daten sind nicht annotiert. Es ist im Rahmen dieser Arbeit nicht möglich, die Annotation durch Expert*innen durchführen zu lassen, weshalb auf das parallel aufgenommene \ac{EKG} zurückgegriffen wird.
