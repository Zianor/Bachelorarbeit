\chapter{Evaluation der Ergebnisse}

\begin{itemize}
	\item Bibliotheken sklearn, XGBoost in eigenem Basismodell
	\item \ac{XGBoost}, da besser als als Implementierung von sklearn \footcite[Kapitel 10]{Harrison2019}
	\item gleiches Testset wie in Kapitel \ref{analyse}
\end{itemize}

\section{Vergleich der Modelle}
	
	\begin{table}[H]
		\begin{tabular}{l l | l|| c | c | c | c }
 						&				& Merkmalsset	& \ac{MAE} [FE]	& Coverage [\%]	& F1-Score	& AUC	\\ \hline
 		insgesamt		&	 			&				& 21{,}85		& -				& - 		& -		\\
 		annotiert		&				&				& 3{,}28			& 43{,}21		& - 		& -		\\ \hline
 		\multirow{4}{*}{Klassifikation}
 						& \multirow{2}{*}{\acs{RF}}		
 										& reduziert 		& 13{,}90		& 32{,}42		& 0{,}54	& 0,69	\\
 						&				& alle			&\\\cline{2-7}
 						& \multirow{2}{*}{\acs{XGBoost}}
 									 	& reduziert		& 14{,}84		& 40{,}97		& 0{,}57	& 0,68	\\ 
 						&				& alle 			&\\\hline
 		\multirow{4}{*}{Regression}				
 					 	& \multirow{2}{*}{\acs{RF}}		
 					 					& reduziert		& 8{,}83			& 12{,}64		& 0{,}34	& 0,67	\\
 					 	&				& alle			&\\\cline{2-7}
 						& \multirow{2}{*}{\acs{XGBoost}}
 										& reduziert		& 12{,}63		& 23{,}12		& 0{,}44	& 0,65	\\
 						&				& alle			&\\
		\end{tabular}
		\caption{Vergleich der aller Modelle mit reduziertem und vollständigem Merkmalsset}
	\end{table}
	
	\begin{itemize}
		\item \ac{XGBoost} höhere Coverage aber auch höheren Fehler
		\item Regression niedrigerer Fehler und niedrigere Coverage
	\end{itemize}
	
	\begin{itemize}
		\item zunächst das in Kapitel \ref{reduction} reduzierte Merkmalsset
		\item Übersicht alle Modelle mit Coverage, MAE, auc auf inf Set
		\item je Regression und Klassifikation um besseres Modell auszuwählen -> Vergleich von XGBoost und RF
		\item Vergleich von Regression und Klassifikation
	\end{itemize}
	
	\section{Evaluation der Merkmale}

	\section{Einfluss der Segmentlänge}

	\section{Einfluss des Schwellwertes der Annotation}