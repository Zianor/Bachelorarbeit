\chapter{Evaluation der Ergebnisse}\label{evaluation}

Im Folgenden werden die entwickelten Modelle untersucht und evaluiert. Für die beiden \acl{RF}s wird die Implementierung von \texttt{sklearn} genutzt, für die Gradient Boosted Trees die Bibliothek \texttt{XGBoost}, da das \ac{XGBoost} besser ist als die in \texttt{sklearn}.\footcite[Kapitel 10]{Harrison2019} Die Ergebnisse der Modelle werden sowohl für das reduzierte als auch das vollständige Merkmalsset berechnet und verglichen. Mit den trainierten Modellen wird erneut eine Merkmalsanalyse durchgeführt. Außerdem wird der Einfluss der gewählten Segmentlänge und des gewählten Schwellwertes der Annotation untersucht. Für alle Untersuchungen wird das gleiche Testset wie in Kapitel \ref{analyse} verwendet.


\section{Vergleich der Modelle}
	
	\begin{table}[H]
		\begin{tabular}{l | l | l|| c | c | c | c }
 						& Merkmalsset	& Modell			& \ac{MAE} [FE]	& Coverage [\%]	& F1-Score	& AUC	\\ \hline
 		\multicolumn{3}{l ||}{insgesamt}					& 21{,}85		& -				& - 		& -		\\
 		\multicolumn{3}{l ||}{annotiert}					& 3{,}28			& 43{,}21		& - 		& -		\\ \hline
 		\multirow{4}{*}{Klassifikation}
 						& \multirow{2}{*}{reduziert}		
 										& \acs{RF} 		& 13{,}90		& 32{,}42		& 0{,}54	& 0,69	\\
 						&				& \acs{XGBoost}	& 14{,}40		& 38,33			& 0{,}56	& 0,68	\\\cline{2-7} 			
 						& \multirow{2}{*}{alle}
 									 	& \acs{RF}		& 11{,}88		& 34,87			& 0{,}6		& 0,75	\\
 						&				& \acs{XGBoost} 	& 11,96			& 41,89			& 0,64		& 0,76\\\hline
 		\multirow{4}{*}{Regression}
 						& \multirow{2}{*}{reduziert}
 										& \acs{RF}		& 8{,}83			& 12{,}64		& 0{,}34	& 0,67	\\
 						&				& \acs{XGBoost}	& 12{,}63		& 23{,}12		& 0{,}44	& 0,65	\\\cline{2-7} 			
 					 	& \multirow{2}{*}{alle}		
 					 					& \acs{RF}		& 8{,}91			& 17,38			& 0{,}44	& 0,74	\\
 					 	&				& \acs{XGBoost} & 10,71			& 30,30			& 0,55		& 0,73\\

 						
		\end{tabular}
		\caption{Vergleich der aller Modelle mit reduziertem und vollständigem Merkmalsset}
	\end{table}
	
	\begin{itemize}
		\item \ac{XGBoost} bei Regression höhere Coverage aber auch höheren Fehler
		\item bei Klassifikation ähnlicher Fehler, aber Coverage von \ac{XGBoost} höher
		\item Regression niedrigerer Fehler und niedrigere Coverage, ähnlich hohe AUC -> mit Verschieben des Schwellwerts näher an Regression?
		\item reduziertes Merkmalsset führt zu schlechteren Ergebnissen -> Feature Importances sollten trotzdem aussagekräftiger sein
	\end{itemize}
	
	\begin{itemize}
		\item zunächst das in Kapitel \ref{reduction} reduzierte Merkmalsset
		\item Übersicht alle Modelle mit Coverage, MAE, auc auf inf Set
		\item je Regression und Klassifikation um besseres Modell auszuwählen -> Vergleich von XGBoost und RF
		\item Vergleich von Regression und Klassifikation
	\end{itemize}
	
	\section{Evaluation der Merkmale}

	\section{Einfluss der Segmentlänge}

	\section{Einfluss des Schwellwertes der Annotation}