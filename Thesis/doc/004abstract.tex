\clearpage
\chapter*{Abstract}\label{abstract}


Ballistokardiographie (BKG) ist eine Messtechnik, bei der die durch den Herzschlag induzierten Massenverschiebungen des Körpers gemessen werden. Störungen, sogenannte Artefakte, die diese Signale überlagern, führen zu Fehlern in der Signalverarbeitung und so womöglich zu fehlerhaften Diagnosen. Die Beurteilung der Signalqualität und damit die zuverlässige Detektion dieser Artefakte ist ein für das BKG bis jetzt nicht hinreichend gelöstes Problem, besonders bei in Betten integrierten BKG-Systemen. 

In dieser Arbeit werden die Grundlagen der Ballistokardiographie erarbeitet - der physiologische Ursprung, die Eigenschaften der Signale und Techniken der Signalverarbeitung. Messdaten werden aufbereitet und Methoden zur Beurteilung der Signalqualität, die sich für andere Aufnahmebedingungen erfolgreich zeigten, evaluiert. Aufbauend auf diesen Ergebnissen werden Merkmale entwickelt, die Informationen über die Signalqualität enthalten. Modelle maschinellen Lernens werden ausgewählt und für Besonderheiten der Daten erweitert.

Abschließend wird die Güte dieser Modelle bewertet und gezeigt, dass die Beurteilung der Signalqualität verbessert werden kann und sich diese Ergebnisse sogar auch auf andere Aufnahmesituationen übertragen lassen.

