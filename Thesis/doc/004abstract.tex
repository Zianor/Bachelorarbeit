\clearpage
\chapter*{Abstract}\label{abstract}

Die Beurteilung der Signalqualität bei ballistokardiographischen Signalen ist für eine Anwendung in der Praxis essentiell, wirft aber besonders bei in Betten aufgenommenen Signalen noch große Fragen auf. Ziel dieser Arbeit ist es, Möglichkeiten der Artefakterkennung bei solchen Signalen mittels maschinellen Lernens zu untersuchen. Grundlage sind bereits entwickelte Verfahren und Fachkenntnis über die Ballistokardiographie 

Schwerpunkte der Arbeit:
\begin{itemize}
	\item Recherche zu verschiedenen Verfahren des Maschinellen Lernens
	\item Beurteilung der Qualität bereits existierender Verfahren für Langzeitaufnahmen von bettlägerigen Patient*innen
	\item Merkmalskonstruktion auf Basis von Domainenexpertise
	\item Vergleich verschiedener Verfahren und Eingabeparamater
	\item Untersuchung des Einflusses der Vorverarbeitung des Signals
	\item Evaluierung und Validierung der Ergebnisse
\end{itemize}

\textbf{Nicht final, nur als Platzhalter}
