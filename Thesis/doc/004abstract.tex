\clearpage
\chapter*{Abstract}\label{abstract}


Ballistokardiographie (BKG) ist eine Messtechnik, bei der die durch den Herzschlag induzierten Massenverschiebungen des Körpers graphisch dargestellt werden. Artefakte, die diese Signale überlagern, führen zu Fehlern in der Signalverarbeitung und so womöglich zu fehlerhaften oder fehlenden Diagnosen. Die Beurteilung der Signalqualität wirft allerdings besonders bei in Betten gewonnenen Langzeitaufnahmen große Fragen auf. 

In dieser Arbeit werden die Grundlagen der Ballistokardiographie erarbeitet - der Hintergrund, die Eigenschaften der Signale und Techniken der Signalverarbeitung. Messdaten werden aufbereitet und Methoden zur Beurteilung der Signalqualität, die sich für andere Aufnahmebedingungen erfolgreich zeigten, evaluiert. Aufbauend auf diesen Ergebnissen werden Merkmale entwickelt, die Informationen über die Signalqualität enthalten. Modelle maschinellen Lernens werden ausgewählt und für Besonderheiten der Daten erweitert.

Abschließend wird die Performance dieser Modelle bewertet und gezeigt, dass die Beurteilung der Signalqualität verbessert werden kann und sich diese Ergebnisse auf Signal aus anderen Aufnahmesituationen übertragen lassen.

