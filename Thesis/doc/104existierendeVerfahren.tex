\chapter{Analyse}\label{analyse}

Im diesem Kapitel werden die betrachteten Daten näher untersucht. Das beinhaltet sowohl die Anwendung der in Kapitel \ref{artefakterkennung} beschriebenen existierenden Verfahren als auch eine Analyse der verwendeten Merkmale.

\section{Aufbau und Evaluation der Verfahren}

Um Modelle zur Beurteilung der Signalqualität anzuwenden und zu untersuchen, müssen sowohl die verwendeten Merkmale extrahiert als auch die Ergebnisse evaluiert werden. In den hier durchgeführten Untersuchungen wurde die Segmentlänge standardmäßig auf 10\,Sekunden festgelegt und der Abstand auf 1\,Sekunde, wodurch sich die Segmente je 90\,\% überlappen. Diese Segmentlänge wird zum einen häufiger verwendet\footcite[Vgl.][]{Yu2020, Sadek2016, Orphanidou2015} und zum anderen bietet sie eine ausreichend große Robustheit für die Annotation der Daten. Die für die Verfahren jeweils benötigten Merkmal werden segmentweise extrahiert und serialisiert. Zusätzlich zu den verwendeten Merkmalen werden allgemeine Informationen über die Segmente gespeichert. Dazu gehören Patient*innen-ID, EKG-Herzrate, BKG-Herzrate, absoluter und relativer Fehler, $E\textsubscript{HR}$ und die binäre Annotation. Für letztere wird als Standard $E\textsubscript{HR} = 10$ als Schwellwert verwendet.

Diese Daten werden anschließend in Trainings- und Validierungsset aufgeteilt. Um diese beiden Gruppen inhaltlich vollständig zu trennen und auszuschließen, dass auf teilweise bekannten Daten validiert wird, geschieht die Trennung anhand der Patient*innen-IDs, sodass Segmente einer Person lediglich in einem der beiden Sets verwendet werden. Das Validierungsset in dieser Arbeit entspricht einem Drittel der Patient*innen, die restlichen zwei Drittel werden zur Datenexploration, zur Merkmalskonstruktion und zum Training verwendet. Die Verteilung der Label und des relativen Fehlers sind in Abbildung %TODO: Figure \ref{}
gezeigt. Von den existierenden Verfahren benötigen zwei keine Trainingsphase, um vergleichbare Ergebnisse zu erhalten, wird auch bei diesen bei der Evaluation nur das Validierungsset betrachtet.

Wie schon in Kapitel \ref{artefakterkennung} beschrieben, sind bei der Verarbeitung von medizinischem Signal zwei Bereiche wichtig: Die \textit{Coverage} und die Aussagekraft des Signals, in diesem Fall also die Genauigkeit der geschätzten Herzrate im Vergleich zur Referenz. Diese beiden müssen gegeneinander aufgewogen werden und werden aus diesem Grund beide betrachtet. Die Qualität der Modelle kann zunächst anhand der binären Klassifikation beurteilt werden. Da diese aber keine Auskunft darüber enthält, wie nah klassifiziertes Signal an dem Threshold für $E\textsubscript{HR}$ liegt, wird letzteres in einer tiefergehenden Evaluation ebenfalls betrachtet. Im Zuge dessen wird der sowohl der Fehler auf dem als informativ klassifizierten Signal betrachtet als auch die \textit{Coverage} in Bezug auf das ganze Signal für verschiedene Fehlergrößen. Insbesondere falsch klassifiziertes Signal, also Falsch-Negative und Falsch-Positive, ist interessant, um zu beurteilen, ob Fehlklassifikationen lediglich im Grenzbereich oder allgemein vorkommen.

\section{Anwendung existierender Verfahren}

Zunächst werden die in Kapitel \ref{artefakterkennung} beschriebenen existierenden Artefakterkennungsverfahren mit den in dieser Arbeit untersuchten Daten wie oben beschrieben getestet und ihre Leistungsfähigkeit untersucht und bewertet.

\subsection{Ähnlichkeit der Intervallschätzer des CLIE-Algorithmus}

Der \ac{SQI}, der die Ähnlichkeit der Intervallschätzer des \ac{CLIE}-Algorithmus angibt, wird im Normalfall herzschlagweise angewendet. Da die Datenannotion nur bereichsweise vorgenommen werden kann, wurde entschieden, ein Segment als informativ zu klassifizieren, wenn mit den Herzschlägen, deren \ac{SQI} über einem Schwellwert $q_{th}$ liegt, eine Coverage von über 80\,\% auf dem Segment erreicht wird. Zum Testen des Algorithmus wurde unter anderem $q_{th} = 0{,}4$ gewählt, da dieser Wert auch von \citeauthor{Zink2017} verwendet wird. Zusätzlich wurden $q_{th} = 0{,}3$ und $q_{th} = 0{,}2$ untersucht, um den Einfluss von $q_{th}$ einzuordnen. Bei der Berechnung der Merkmale werden für jedes Segment die detektierten Herzschläge extrahiert, deren \ac{SQI} über $q_{th}$ liegt. Auf Basis dieser Intervalllängen wird wie in Kapitel \ref{annotation} beschrieben die Herzrate, im Folgenden $HR\textsubscript{SQI}$ genannt, ermittelt. Extrahierte Merkmale sind damit in diesem Fall $HR\textsubscript{SQI}$ und die Coverage $C\textsubscript{SQI}$.

 % q_{th}=0{,}4
 \begin{figure}
 	\centering
 	\begin{tabular}{l || c | c | c }
 					& insgesamt	& annotiert		& klassifiziert	\\ \hline
 		\ac{MAE} 	& 20{,}18	& 3{,}18			& 11{,}48		\\
 		Coverage	& -			& 43{,}22\,\%	& 10{,}46\,\%	\\
 	\end{tabular}
 	\caption[Fehler und Coverage der Klassifikation nach der Ähnlichkeit der Intervallschätzer des CLIE-Algorithmus mit $q_{th}=0{,}4$ im Vergleich zum gesamten Signal und der Annotation]{Fehler und Coverage der Klassifikation nach der Ähnlichkeit der Intervallschätzer des CLIE-Algorithmus mit $q_{th}=0{,}4$ im Vergleich zum gesamten Signal und der Annotation}
 \end{figure}

\begin{itemize}
	\item sehr gute Genauigkeit der erkannten Herzschläge
	\item trotzdem knapp 22\,\% der informativen Segmente mit $E\textsubscript{HR} > 20$
	\item keine unbrauchbaren Segment als informativ gelabelt
	\item sehr geringe Coverage mit 9{,}27\,\% auf Testset im Vergleich zu 43{,}22 der Annotation
\end{itemize}


\subsection{Schwellwerte für Standardabweichung, Minimum und Maximum}

Für dieses Verfahren werden lediglich zwei Schwellwerte $T_1$ und $T_2$ benötigt, die auf Standardabweichung, Minimum, Maximum und Durchschnitt des Signals beruhen. Die Klassifikation ist ein einfacher Vergleich. \citeauthor{Pino2015} verwenden sehr kleine Fenster, die hier aufgrund Robustheit der Annotation nicht verwendet werden können. Stattdessen wird der Algorithmus sowohl auf 10\,Sekunden- als auch 4\,Sekunden-Segmenten getestet, damit sichtbar wird, ob die Segmentlänge einen Einfluss auf das Ergebnis hat.

Der Test mit beiden Segmentlängen zeigt, dass dieses Verfahren nicht nutzbar ist, da bei beiden über 95\,\% der Daten als informativ klassifiziert werden, darunter auch Segmente, bei denen $E\textsubscript{HR}$ maximal ist. Weiterführende Evaluation bietet hier keine weiteren Erkenntnisse. Dieses Verfahren ist damit nicht weiter nutzbar.

\subsection{Maschinelles Lernen mittels statistischer Merkmale}

Für das maschinellen Lernen mittels statistischer Merkmale werden die in Kapital \ref{ml-beschreibung} aufgezählten Merkmale extrahiert. Als Bibliothek für die Modelle des maschinellen Lernes wird \textit{scikit-learn}\footcite[]{scikit-learn} verwendet. Da die Daten sich grundlegend von den von \citeauthor{Sadek2016} untersuchten unterscheidet, werden die Hyperparameter der Modelle im Zuge dieser Arbeit erneut optimiert. Bei der dafür durchgeführten Kreuzvalidierung wird wie schon bei der Unterteilung in Trainings- und Validierungsset anhand der Patient*innen-ID geteilt, damit auch diese aussagekräftige Ergebnisse liefert.

\section{Datenanalyse und Merkmalskonstruktion}


