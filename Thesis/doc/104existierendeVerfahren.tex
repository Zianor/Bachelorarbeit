\chapter{Existierende Verfahren zur Beurteilung der Signalqualität}\label{existierende}

\begin{itemize}
	\item gibt einige Verfahren
	\item Vermutung, dass nicht hinreichend für unsere Daten	 aus Gründen
\end{itemize}

\section{Ähnlichkeit der Intervallschätzer nach Brüser}

\section{Schwellwerte für Standardabweichung, Minimum und Maximum}

\section{Maschinelles Lernen mittels statistischer Merkmale}

Ein Algorithmus zur Beurteilung der Signalqualität mittels maschinellen Lernens wird von \citeauthor{Sadek2016} im Paper \citetitle{Sadek2016} beschrieben. Betrachtet werden \ac{BKG}-Signale, die in einem Massagesessel aufgenommen werden.

\subsection{Beschriebenes Vorgehen}

\subsection{Eigene Implementierung}

\begin{itemize}
	\item Hyperparameter selbst ermittelt, da anderer Anwendungszweck
\end{itemize}

\subsection{Ergebnisse im Vergleich}

\subsection{Evaluation}


